\documentclass{amsart}   
\usepackage{RepStyle} 

\usepackage{CJKutf8}

\begin{document}

\title{The Two-Column Specht Module of the Hecke Algebra of $S_n$}
\author{Miles Johnson \& Natalie Stewart}
\maketitle

\section{Introduction}
Let $S_{2n}$ be the symmetric group on $2n$ indices, let $\SH = \SH_{k,q}(S_{2n})$ be the corresponding Hecke algebra with parameter $p$, and let $S = \cbr{T_1,\dots,T_{n-1}}$ be the simple transpositions generating $\SH$.
Let $V := S^{(2,\dots,2)}$ be the Specht module corresponding to the young diagram with rows of length 2.
The purpose of this writing is to characterize this representation via an isomorphism with another representation of $\SH$.
\begin{definition}
  A \emph{crossingless matching} on $2n$ indices is a partition of $\cbr{1,\dots,2n}$ into parts of size $2$ such that no two parts ``cross'', i.e. there are no parts $(a,a')$ and $(b,b')$ such that $a < b < a' < b'$.
  Then, define $W_{2n}$ to be the $\CC$-vector space with basis the set of crossingless matchings on 2n indices.

  In order for this to be a $\SH$-module, endow this with the action given by Figure \ref{Action}; if this creates a loop, simply scale by $(1 + q)$, and otherwise deform into a crossingless matching and scale by $q^{1/2}$.
\end{definition}

\begin{figure}
  \[
    \Action{6}{1/4, 2/3, 5/6}{4}{1/6, 2/3, 4/5}{q^{1/2}}
    \hspace{50pt}
    \Action{6}{1/4, 2/3, 5/6}{2}{1/4, 2/3, 5/6}{(1 + q)}
  \]
  \caption{Illustration of the actions $(1 + T_4)w_3$ and $(1 + T_2)w_3$ in $W_6$.
  In general, we act on basis elements by simple transpositions by deleting loops, deforming into a crossingless matching, and scaling based on whether a loop was deleted.}
  \label{Action}
\end{figure}

Let the length of an arc $(i,j)$ be $l(i,j) := j - i + 1$.
Note that the crossingless matchings can all be identified with a list of $n$ integers describing the lengths of the arcs from left to right;
using this, we may order the crossingless matchings in increasing lexicographical order in order to obtain an order on the basis.
Let $C_n$ be the $n$th catalan number, and let the resulting basis for $W_{2n}$ be $\cbr{w_1,\dots,w_{C_n}}$ as illustrated in Figure \ref{S6 Basis}.

\begin{figure}
  \def\cbasisspacing{5mm}
  $\cbr{
    \begin{gathered}
      \Matching{6}{1/2, 3/4, 5/6}, \hspace{\cbasisspacing}
      \Matching{6}{1/2, 3/6, 4/5}, \hspace{\cbasisspacing}
      \Matching{6}{1/4, 2/3, 5/6}, \hspace{\cbasisspacing}
      \Matching{6}{1/6, 2/3, 4/5}, \hspace{\cbasisspacing}
      \Matching{6}{1/6, 2/5, 3/4}
    \end{gathered}}$
    \caption{The increasing lexicgraphical basis for $W_{6}$.}
  \label{S6 Basis}
\end{figure}

We will prove that $W := W_{2n}$ is isomorphic to $V$ as representations in the case that $\SH$ is semisimple.
To do so, we will prove that $W$ has an irreducible restriction to $S_{2n-1} \subset S_{2n}$;
using the branching theorem, this implies that $W$ is isomorphic to a Specht module corresponding to a rectangular young diagram.

We will move on to prove that these modules have unique dimension up to transposition of the diagram;
then, we will show that $\dim V = \dim W$ so that $W$ corresponds to an $n \times 2$ or $2 \times n$ diagram.
We will then do a short character computation to prove that $V \cong W$.

\newpage
\section{Irreducibility} 
We can now begin by proving that $\Res_{\SH_{k,p}(S_{2n - 1})}^{\SH_{k,p}(S_{2n})} W$ is irreducible;
in partuclar, this implies that $W$ itself is irreducible, as an $\SH$-subrepresentation is a $\SH_{k,p}(S_{2n - 1})$-subrepresentation.
\begin{proposition}
  Set $\SH' := \SH_{k,p}(S_{2n - 2}) \subset \SH$.
  Then, $\Res_{\SH'}^{\SH} W$ is irreducible if $q$ is generic or a primitive $e$'th root of unity for $e \geq n + 1$ or $e = 2$.
\end{proposition}
\begin{proof}
  We will prove the equivalent condition that each vector in $w \in W$ is \emph{cyclic}, i.e. $\SH' w = W$.

  We will first prove that $w_1$ is cyclic, for which it is sufficient to prove that every basis vector of $W$ is in $Aw_1$.
  fix wome basis vector $w_k$, and suppose that it contains arc $(1,j)$,
  Then, the vector
  \[
    w' := (1 + T_2)(1 + T_4)\dots(1 + T_{j-2})w_1
  \]
  contains an arc $(1,j)$ and all other arcs are of the form $(a,a+1)$ for some $a$.
  We may separately act on the subset of arcs with $1 < a < j$ and with $a > j$;
  this process gave our base case of $W_4$, and allows us to recurse to $W_{2m}$ with $m < n$, outlining explicit vectors $h \in \SH'$ with $hw_1 = w_i$.
  Hence it is sufficient to prove that $w_1$ is generated by every $w \neq 0 \in W$.
  
  Since the image of $(1 + T_i)$ has arc $(i,i+1)$, it is isomorphic as a vector space to $W_{2n}$ with $2n - 2$ vertices.\footnote{This ismorphism ``ignores'' the arc $(i,i+1)$.}
  This isomorphism carries $w_1$ to an equivalent vector $w'_1$ in $W_{2n - 2}$ containing all length-2 arcs.
  Further, all actions of $(1 + T'_j) \in \SH_{k,p}(S_{2n - 4})$ act identically (through the isomorphism) with simple transpositions other than $1 + T'_i$, which acts equivalently to $q^{-1}(1 + T_i)(1 + T_{i + 1})(1 + T_{i - 1})$ as illustrated in Figure \ref{bigloop};
  hence, if the image of $(1 + T_i)w$ in $W_{2n - 2}$ generates an ideal containing $w'_1$, then $(1 + T_i)w$ generates an ideal containing $w_1$, and hence $w$ is cyclic. 

  \begin{figure}[b]
  \[
    \Action{6}{1/6, 2/5, 3/4}{2}{1/6, 2/3, 4/5 }{q^{1/2}} \hspace{20pt} 
    \IsoAction{8}{1/8, 2/7, 3/4, 5/6}{3}{1/8, 2/5, 3/4, 6/7}{q^{1/2}}
  \]
  \[
    \Action{4}{1/4, 2/3}{2}{1/4,2/3}{(1 + q)} \hspace{20pt}
    \IsoAction{6}{1/6, 2/5, 3/4}{3}{1/6, 2/5, 3/4}{(1 + q)}
  \]
  \caption{The correspondence between the action of $(1 + T_2)$ on $w'_5 \in W_6$ and the action of $q^{-1}(1 + T_3)(1 + T_4)(1 + T_2)$ on the corresponding vector in $W_8$ having arc $(3,4)$ first, then on $w'_2 \in W_4$.
  This demonstrates that the action works with and without creating a loop.
  }
  \label{bigloop}
  \end{figure}

  We are now ready to make the central claim in our proof:
  \begin{claim*}
    Suppose $n > 1$.
    Then, the intersection $K := \bigcap_{i = 1}^{2n - 2} \ker (1 + T_i)$ is trivial if and only if $e \neq n + 1$.
  \end{claim*}
  This claim is necessary for irreducibility, as $K$ is a proper subrepresentation of $W$.
  
  Suppose this claim is true.
  We will use induction on $n$ to prove irreducibility;
  the base case $n = 1$ is clear, so suppose $W_{2n - 2}$ is irreducible for all $e \geq n$ or $e = 2$, and pick some $w \in W_{2n}$.
  Then, pick some $1 + T_i$ such that $(1 + T_i)w \neq 0$, and pick some action $h' \in W_{2n - 2}$ which takes the image of $(1 + T_i)w$ in $W_{2n - 2}$ to $w'_1$;
  this pulls back to an action $h \in W_{2n}$ such that $h(1 + T_i)w = w_1$, so $W_{2n}$ is irreducible.

 \textit{Proof of claim.}
 \newcommand{\textoverline}{$\overline{\mbox{\phantom{L}}}$}
 \begin{CJK}{UTF8}{min}
 \textoverline\verb|\_(ツ)_/|\textoverline
\end{CJK}
\end{proof}

Suppose $\SH$ is semisimple;
then $W$ is a Specht module corresponding to a rectangular region.
We may now use dimension, for which one should note that there is an easy bijection between standard tableaus on $n = 2 + \dots + 2$ and the Dyck paths on $2n$ points by specifying that the number in the $(i,1)$ box of the tableau is the index of the $i$th ``up'' path.
Hence $V_{(2,\dots,2)}$ has dimension the $n$th catalan number $C_n$.

Similarly, we may biject the Dyck paths on $2n$ points with the crossingless matchings, by making the value at point $i$ the number of ``open crossings'' at that point, i.e. the number of arcs $(a,b)$ with $a \leq i < b$.
Hence $V$ and $W$ have the same dimension.
This pins the shape of the diagram corresponding to $W$ as follows.

\begin{proposition}
  Let $V_1$ and $V_2$ be two specht modules corresponding to $a_1 \times b_1$ and $a_2 \times b_2$ rectangular young diagrams.
  Then, $\dim V_1 = \dim V_2$ if and only if the diagram of $V_1$ is the same as or a transposition of the diagram of $V_2$.
\end{proposition}
\begin{proof}
  Recall that, if they have the same diagram then $V_1 \cong V_2$, and if they're transposed from each other, $V_1 \cong V_2 \otimes U$ where $U$ is the alternating representation;
  hence $\dim V_1 = \dim V_2 \cdot \dim U = \dim V_2$.

  Suppose WLOG that $a_1 < a_2 < b_2 < b_1$.
  By the hook-length formula, it is sufficient to give a bijective correspondence between the boxes in $V_2$ and $V_1$ such that the hook-length in $V_2$ is larger than the hool-length in $V_1$, and at least once strictly larger.
  We can give this correspondence by listing the boxes beginning at the bottom right corner, then increasing up left-to-right diagonals, and note that this satisfies our conditions.
\end{proof}

Let $V' := V_{(n,n)}$.
Then, $W$ is isomorphic to exactly one of $V$ and $V'$.
It is hence sufficient to prove that $W$ is not isomorphic to $V'$, and we may do this via a character computation.


\end{document}
